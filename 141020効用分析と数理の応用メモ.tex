\documentclass[a4j,12pt]{jsarticle}

%---------------preamble
\usepackage{array,booktabs,newpxmath}
\usepackage[top=1.5cm, bottom=1.5cm, left=1.5cm, right=1.5cm, includefoot]{geometry}
\usepackage[dvipdfmx]{graphicx}
\title{効用分析の数理と応用メモ}
\author{}
\date{\today}
%---------------preamble

\begin{document}
\maketitle


\section{期待効用仮説}
期待効用仮説は、複数ある案の中から、もっとも良い案を選出するうえでの一つの考え方(仮説)である.期待効用仮説に基づけば、意思決定者は、代替案の集合、結果の集合、結果の確率をもとに、効用の期待値(期待効用)が最大となる代替案を選択する.

\begin{center}
\begin{tabular}{ll} 
代替案の集合 & $A=\{a_1,a_2,\dots,a_j,\dots,a_m\}$ \\
結果の集合 & $X=\{x_1,x_2,\dots,x_i,\dots,x_n\}$ \\
代替案$j$の結果が生起する確率 & $\{p_1^j,p_2^j,\dots,p_i^j,\dots,p_n^j\}$ \\
結果の効用 & $\{u(x_1),u(x_2),\dots,u(x_i),\dots,u(x_n)\}$ \\
代替案$j$期待効用 & $\sum_{i=1}^n p_i^j u(x_i)$
\end{tabular}
\end{center}

\subsection{例1}
一例として、降水確率40\%の状況下で傘をもつかどうかの問題を考える.表1では、二つの代替案$\{ 傘をもつ,傘をもたない\}$と、これより得られる三つの結果$\{ 雨に濡れる,雨に濡れないが傘をもつ,雨に濡れず傘をもたない\}$があり、期待値が最大化する代替案は「a.傘をもつ」となる.


\begin{table}[htbp]
\begin{center}
\begin{tabular}{ccccc}  \toprule
	代替案		 	& 確率		& 結果 							& 効用 				& 期待効用					 	\\ 
	$a_j\in A$			 	& $p_i^j$		& $x_i \in X$ 							& $u(x_i)$ 			& $ \sum_{i=1}^3 p_i^j u(x_i)$ 	\\ 	\midrule
					& $p_1^1:\ \ 0\%$  		& $x_1:$雨に濡れる 					& 0.0 				& 							\\ 		
	$a_1:$傘をもつ 	& $p_2^1:100\%$ 	& $x_2:$雨に濡れないが、傘をもつ 		& 0.7				& $1.0\times0.7=0.7$			\\ 				
					& $p_3^1:\ \ 0\%$  		& $x_3:$雨に濡れず、傘をもたない 		& 1.0 				& 							\\	\midrule
					& $p_1^2:\ 40\%$  		& $x_1:$雨に濡れる 					& 0.0 				& 							\\ 		
	$a_2:$傘をもたない 	& $p_2^2:\ \ 0\%$ 		& $x_2:$雨に濡れないが、傘をもつ 		& 0.7 				& $0.6\times1.0=0.6$			\\ 		
					& $p_3^2:\ 60\%$  		& $x_3:$雨に濡れず、傘をもたない 		& 1.0 				& 							\\	\bottomrule
\end{tabular}
\end{center}
\caption{代替案、確率、結果、効用、期待効用の関係}
\end{table}

\newpage

\section{くじと確実同値額}
期待効用仮説に基づけば、意思決定者は期待効用が最大となる代替案を選択する.
表1において明らかなように、代替案とは、結果と結果の生起する確率の対応関係によって示される.この対応関係を「くじ」と考えれば、
意思決定者は、期待効用が最大となる「くじ」の選択をすると考えることができる.
「くじ」$l$を、

\begin{center}
$l_{a_1}=(x_1,x_2,x_3 ; p_1^1,p_2^1,p_3^1)$ \\
$l_{a_2}=(x_1,x_2,x_3 ; p_1^2,p_2^2,p_3^2)$
\end{center}

とする.また、結果を選択することと、くじを選択することは以下のように表すことができる.


\begin{center}
\begin{tabular}{ll} 	
結果の効用 & $u(x_i)$ \\
くじの効用=期待効用 & $\displaystyle u(l_{a_1})=\sum_{i=1}^3 p_i^1 u(x_i)$ \\
\end{tabular}
\end{center}

\subsection{例2}
いま、表1における降水確率が30\%となった状況(表2)を考えてみる.このとき、くじlaを選択するときの期待効用と、結果$x_2$が得られる効用は一致するため、以下の等式が得られる.このような、くじの期待効用に等しい結果の効用を、「確実同値額」という.

\begin{center}
$\displaystyle u(l_{a_2})=u(x_2)$ \\
\end{center}

\begin{table}[b]

\begin{center}
\begin{tabular}{ccccc} 																								\toprule
	代替案		 	& 確率		& 結果 							& 効用 				& 期待効用					 	\\ 
	$a_j\in A$			 	& $p_i^j$		& $x_i \in X$ 							& $u(x_i)$ 			& $\displaystyle \sum_{i=1}^3 p_i^j u(x_i)$ 	\\ 	\midrule
					& $p_1^1:\ \ 0\%$  		& $x_1:$雨に濡れる 					& 0.0 				& 							\\ 		
	$a_1:$傘をもつ 	& $p_2^1:100\%$ 	& $x_2:$雨に濡れないが、傘をもつ 		& 0.7				& $1.0\times0.7=0.7$			\\ 				
					& $p_3^1:\ \ 0\%$  		& $x_3:$雨に濡れず、傘をもたない 		& 1.0 				& 							\\	\midrule
					& $p_1^2:\ 30\%$  		& $x_1:$雨に濡れる 					& 0.0 				& 							\\ 		
	$a_2:$傘をもたない 	& $p_2^2:\ \ 0\%$ 		& $x_2:$雨に濡れないが、傘をもつ 		& 0.7 				& $0.7\times1.0=0.7$			\\ 		
					& $p_3^2:\ 70\%$  		& $x_3:$雨に濡れず、傘をもたない 		& 1.0 				& 							\\	\bottomrule\end{tabular}
\end{center}
\caption{代替案、確率、結果、効用、期待効用の関係}
\end{table}


\subsection{例3.アンケートによる確実同値額の同定法}
期待効用仮説に従えば、確実同値額は結果の効用の期待値と一致するが、実際の確実同値額はそうはならないことがある。
期待効用と、確実同値額の差は、リスクに対する態度を表していると考えられており、確実同値額の同定によって、意思決定者のリスクについての態度を確認することができる。以下は確実同値額を同定するためのアンケートの例である。確率が五分五分で起こるくじと、確実同値額を比較させて回答を得る。

\begin{table}[h]
\begin{center}
\begin{tabular}{ll} 																	\toprule
	質問1		 	& 1,000円と10,000円が50\%-50\%の確率で入っているくじ$l_{2}$がある。	\\
                                & このくじと同じ程度の満足度を得られる金額$x_{0.5}$はいくらか			\\ \midrule
	質問2		 	& 1,000円と金額$x_{0.5}$が50\%-50\%の確率で入っているくじ$l_{1}$がある。 	\\
					& このくじと同じ程度の満足度を得られる金額$x_{0.25}$はいくらか			\\ \midrule
	質問3		 	& 金額$x_{0.5}$と10,000円が50\%-50\%の確率で入っているくじ$l_{3}$がある。	\\
					& このくじと同じ程度の満足度を得られる金額$x_{0.75}$はいくらか			\\ 	\bottomrule


\end{tabular}
\end{center}
\caption{確実同値額を求めるアンケート}
\end{table}

このアンケートでは、以下の対応を示す各xを求めることができる。
\begin{center}

$u(l_1)\sim u(x_{0.25})$ \\
$u(l_2)\sim u(x_{0.5})$ \\
$u(l_3)\sim u(x_{0.75})$

\end{center}

ここで、期待効用仮説に基づく合理的な回答者であれば、以下の回答が得られる。


\begin{center}

$u(l_2)=0.5\times 1,000 + 0.5 \times 10,000 = 5,500$ \\
$u(l_1)=0.5\times 1,000 + 0.5 \times 5,500 = 2,750$ \\
$u(l_3)=0.5\times 5,500 + 0.5 \times 10,000 = 7,750$

\end{center}

一方、実際のアンケート結果が以下であったとする。

\begin{center}

$x_{0.5}=5,500$ \\
$x_{0.25}=3,000$ \\
$x_{0.75}=7,000$ \\

\end{center}

このときの、$x_{0.25}$と$x_{0.75}$において、期待効用と異なる結果となっているが、

\begin{center}
\includegraphics[width=10cm]{kakujitsu2.png}
\end{center}

\subsection{例4.アンケートによる確実同値額の同定法2}

確実同値額はアンケートの取り方によって変わる。先の質問を変更し、

\section{その他メモ}

確実同値額、リスクに対する態度(リスク回避型)などを説明しやすい具体例を検討中$\dots$



\begin{thebibliography}{99}
\bibitem{test} 田村,中村,藤田「効用分析の数理と応用」コロナ社1997年
\bibitem{syouhin} 岡本「商品選択問題についての多属性効用関数法の応用」経営情報科学 vol.2 No.3 1989年
%\bibitem{2}中山,谷野「多目的計画法の理論と応用」コロナ社1994年
\end{thebibliography}

\end{document}